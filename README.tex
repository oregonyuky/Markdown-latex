\documentclass[15pt]{article}
\usepackage{fancyhdr}
\usepackage{amsmath}
\usepackage{fancyvrb}
\usepackage{tikz}
\usepackage{pgfplots}
\pgfplotsset{compat=1.11}
\pagestyle{fancy}
\fancyhead{}
\fancyfoot{}
\fancyhead[L]{Geometria Analitica}
\fancyhead[R]{Felipe Akio Nishimura}
\fancyfoot[C]{\thepage}
\usepackage{graphicx}

\begin{document}
\title{Geometria Analitica}
\author{Felipe Akio Nishimura}

\maketitle
\textbf{Matrizes e seus variados}\\
\\
\\
\\
\textbf{A imagem abaixo representa uma matriz 2x3:}
\\
\\
\centerline{
$\begin{matrix}
A_{2\times3}=
\begin{bmatrix}
a_1 & b_1 & c_1\\
a_2 & b_2 & c_2\\
\end{bmatrix}
\end{matrix}$
}
\\
\\
\textbf{Matriz transposta:}\\
\\
\centerline{
$A=(a_{ij})_{m\times n} \Rightarrow A^T=(a_{ji})_{n\times m}$
}
\\
\\
\centerline{
$\begin{matrix}
A=
\begin{bmatrix}
2 & -1 & 1\\
3 & 0 & -5\\
\end{bmatrix}
\end{matrix}
\Rightarrow
\begin{matrix}
A^T=
\begin{bmatrix}
2 & 3\\
-1 & 0\\
1 & -5\\
\end{bmatrix}
\end{matrix}$
}\\
\\
\textbf{Matriz nula:}\\
\\
\centerline{
$\begin{matrix}
O_{2\times 3}=
\begin{bmatrix}
0 & 0 & 0\\
0 & 0 & 0\\
\end{bmatrix}
\end{matrix}$
ou
$\begin{matrix}
O_{2\times 2}=
\begin{bmatrix}
0 & 0\\
0 & 0\\
\end{bmatrix}
\end{matrix}$
}\\
\\
\textbf{Matriz oposta:}\\
\\
\centerline{
$\begin{matrix}
A=
\begin{bmatrix}
0 & 3\\
-2 & 5\\
\end{bmatrix}
\end{matrix}
\Rightarrow
\begin{matrix}
-A=
\begin{bmatrix}
0 & -3\\
2 & -5\\
\end{bmatrix}
\end{matrix}$
}\\
\\
\textbf{Matriz quadrada:}
\\
\\
\centerline{
$\begin{matrix}
\begin{bmatrix}
0 & 3\\
-2 & 5\\
\end{bmatrix}
\end{matrix}$
é uma matriz quadrada de ordem 2.
}\\\\\\
\centerline{
$\begin{matrix}
\begin{bmatrix}
3 & 0 & -3\\
7 & -2 & 5\\
1 & 4 & 0\\
\end{bmatrix}
\end{matrix}$
é uma matriz quadrada de ordem 3.
}
\\
\\\\
\textbf{Diagonal principal:}
O conjunto dos elementos $a_{ij}$ em que $i=j$
\\
\[
  \begin{bmatrix}
    d_{11} & & \\
    & \ddots & \\
    & & d_{33}
  \end{bmatrix}
\]
\textbf{Diagonal secundaria:}
O conjutno dos elemnetos $a_{ij}$ em que $i+j=n+1$

\[
\begin{bmatrix}
&  & d_{13} \\
& \ddots &  \\
d_{31} & &  
\end{bmatrix}
\]
\textbf{Matriz diagonal:}
\\
Toda matriz quadrada em que todos os elementos fora da diagonal principal são principal são iguais a zero 
\\\\
\centerline{
$\begin{matrix}
M=
\begin{bmatrix}
3 & 0\\
0 & -5\\
\end{bmatrix}
N=
\begin{bmatrix}
1/2 & 0 & 0\\
0 & 0 & 0\\
0 & 0 & 2
\end{bmatrix}
\end{matrix}$
}
\\
\\
\textbf{Traço de uma matriz quadrada:}\\
A soma dos elementos de sua diagonal principal
\\\\
\centerline{
$\begin{matrix}
M=
\begin{bmatrix}
3 & 0\\
0 & -5\\
\end{bmatrix}
N=
\begin{bmatrix}
1/2 & 0 & 0\\
0 & 0 & 0\\
0 & 0 & 2
\end{bmatrix}
\end{matrix}$
}
\\\\
\centerline{
Traço de $M$ é -2
}
\centerline{
Traço de $N$ é 3/2
}
\end{document}
