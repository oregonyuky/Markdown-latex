\documentclass[15pt]{article}
\usepackage{fancyhdr}
\usepackage{amsmath}
\usepackage{fancyvrb}
\usepackage{tikz}
\usepackage{pgfplots}
\pgfplotsset{compat=1.11}
\pagestyle{fancy}
\fancyhead{}
\fancyfoot{}
\fancyhead[L]{MA401}
\fancyhead[R]{Date: JAN/12/2018}
\fancyfoot[C]{\thepage}
\usepackage{graphicx}

\begin{document}
\title{LECTURE 08}
\author{KUMAR SHUBHAM (16CS08)}

\maketitle
\textbf{SYSTEM OF LINEAR EQUATION}\\
\\
\\
\\
\textbf{General form of system of linear equation with 3 variables}\\

\[
\left\{ 
\begin{array}{c}
a_1x+b_1y+c_1z=d_1 \\ 
a_2x+b_2y+c_2z=d_2 \\ 
a_3x+b_3y+c_3z=d_3
\end{array}
\right. 
\]
\\ 
\\
\textbf{A Matrix Method to Solve a System of n Linear Equations in n unknowns:}\\
\centerline{
\textbf{A \space X \space = \space D}}
\\
\\
\\
\centerline{
$\begin{matrix}
A=
\begin{bmatrix}
a_1 & b_1 & c_1\\
a_2 & b_2 & c_2\\
a_3 & b_3 & c_3\\
\end{bmatrix}
\end{matrix}$
$\begin{matrix}
 X=
\begin{bmatrix}
x_1\\
x_2\\
x_3\\
\end{bmatrix}
\end{matrix}$
$\begin{matrix}
D=
\begin{bmatrix}
d_1\\
d_2\\
d_3\\
\end{bmatrix}
\end{matrix}$
}\\
\\
\\
\textbf{Write the augmented matrix that represents the system.}\\
\[
 \left[\begin{array}{rrr|r}
a_1 & b_1 & c_1 & d_1\\
a_2 & b_2 & c_2 & d_2\\
a_3 & b_3 & c_3 & d_3\\
   \end{array} \right]
\]\\
\\
\textbf{Perform row operations to simplify the augmented matrix to one having zeros below the diagonal of the coefficient portion of the matrix.\\
\\
(An entry is on the diagonal of the coefficient
portion of the matrix if it is located in row i and
column i for some positive integer i ≤ n.)\\
\\
If the augmented matrix is equivalent to a matrix
with zeros below the diagonal and all non-zero
entries on the diagonal, then the corresponding
system has a unique solution. \\
\\
If the aumented matrix is equivalent to a matrix
with zeros below the diagonal and at least one
zero on the diagonal, then the corresponding
system does not have a unique solution. \\
\\
In this case, examination of the rows which contain
a zero on the diagonal entry will determine
whether the corresponding system has no solution
or an infinite number of solutions.
}\\
\\
\[
\left[\begin{array}{rrr|r}
  0 &  4 & -4 & 8\\
  2 & -3 &  2 & 1\\
  0 &  0 &  0 & 1\\
  \end{array}\right]
  \]\\
  \\
\textbf{The system of equations represented by the following
augmented matrix has no solution.The third row of the above matrix represents
the equation: 0x + 0y + 0z = -6 or 0 = -6 which is not
a true statement. Therefore the corresponding system
has no solution.}\\
\\
\[
\left[\begin{array}{rrr|r}
  0 &  4 & -4 & 8\\
  2 & -3 &  2 & 1\\
  0 &  0 &  0 & 0\\
  \end{array}\right]
  \]\\
  \\
\textbf{The system of equations represented by the following
augmented matrix has an infinite number of solutions. Third row of the above matrix represents
the equation: 0x + 0y + 0z = 0 or 0 = 0 which is a true for any values x,y and z.}\\
\\
\\
\textbf{Some sample problems are here:-}\\
\\
\textbf{Example 1 :: Solve the following system of linear equations :- }
\[
\left\{ 
\begin{array}{c}
x_2-4x_3=8 \\ 
2x_1-3x_2+2x_3=1 \\ 
4x_1-8x_2+12x_3=1
\end{array}
\right. 
\]
\textbf{Solution :}\\
\\
\hspace{1.5cm}
\textbf{Step 1 : Write the matrix form of the equation }\\
\\
\centerline{
$\begin{matrix}
\begin{bmatrix}
0 &  4 & -4\\
2 & -3 &  2\\
4 & -8 & 12\\
\end{bmatrix}
\end{matrix}$
$\begin{matrix}
\begin{bmatrix}
x_1\\
x_2\\
x_3\\
\end{bmatrix}
\end{matrix}$
$\begin{matrix}
\space = \space
\begin{bmatrix}
8\\
1\\
1\\
\end{bmatrix}
\end{matrix}$}
\\ 
\\ 
\\
\centerline{
$\begin{matrix}
X=
\begin{bmatrix}
x_1\\
x_2\\
x_3\\
\end{bmatrix}
\end{matrix}$}
\\
\\
\\
\textbf{Step 2 : Convert it into agumented matrix }\\
\\

\[
 \left[\begin{array}{rrr|r}
  0 &  4 & -4 & 8 \\
  2 & -3 &  2 & 1 \\
  4 & -8 & 12 & 1 \\ 
   \end{array} \right]
\]\\
\\
\textbf{Step 3 : Solve it using various row elementary operations }\\
\\
\\
\[
\left[\begin{array}{rrr|r}
  0 &  4 & -4 & 8 \\
  2 & -3 &  2 & 1\\
  4 & -8 & 12 & 1  \\
  \end{array}\right]
  \]
  \\
  \\
  \textbf{Our strategy is to use elementary row operations to zero out the entries in: \\ Row 2 and Column 1\\ Row 3 and Column 1 \\ Row 3 and Column 2.}
  \\
  \\
  \[
  \xrightarrow{\substack{R_3 \rightarrow R_3 - 2R_2}}
  \left[\begin{array}{rrr|r}
0 &  4& -4 & 8 \\
2 & -3 & 2 & 1 \\
4 & -8 & 12 & 1 \\
  
  \end{array}\right]
  \]
  \\
  \\
  \
  \[
 \xrightarrow{\substack{R_2 \leftrightarrow R_1}}
\left[\begin{array}{rrr|r}
  2 & -3& 2 & 1 \\
  0 & 4 & -4 & 8 \\
  0 & -2 & 8 & -1 \\
  \end{array}\right]
  \]\\
  \\
  \[
 \xrightarrow{\substack{R_1 \rightarrow R_1/2\\R_2\rightarrow R_2/4\\R_3\rightarrow R_3/2}}
\left[\begin{array}{rrr|r}
  1 & -3/2& 1 & 1/2 \\
  0 & 1 & -1 & 2 \\
  0 & -1 & 4 & -1/2 \\
  \end{array}\right]
  \]\\
  \\
  \\
   \textbf{HINT : When all entries of a row have a common factor, consider dividing each term in that row by the common factor. If you can reduce the magnitude of the entries in a row without introducing fractions, your subsequent calculations will involve smaller numbers}
   \\
   \\
   \[
 \xrightarrow{\substack{R_1 \rightarrow R_1 + 3/2 R_2\\ R_3\rightarrow R_3 + R_2 \\ R_3 \rightarrow R_3/3}}
\left[\begin{array}{rrr|r}
  1 & 0 & -1/2 & 7/2 \\
  0 & 1 & -1 & 2 \\
  0 & 0 & 1 & 1/2 \\
  \end{array}\right]
  \]\\
  \\
  \\
  
 \[
 \xrightarrow{\substack{R_1 \rightarrow R_1 + 1/2 R_2\\ R_2\rightarrow R_2 + R_3}}
\left[\begin{array}{rrr|r}
  1 & 0 & 0 & 15/4 \\
  0 & 1 & 0 & 5/2 \\
  0 & 0 & 1 & 1/2 \\
  \end{array}\right]
  \]\\
\textbf{Step 4 :Equate each row  }\\
\\
\textrm{}{Here rank of agumented matrix is same as compared to original matrix , therefore it has unique solution.}
\\
\\
\textbf{Solution is }
\\
\centerline{$ x_1 = 15/4 $}\\
\centerline{$ x_2 = 5/2 $}\\
\centerline{$  x_3 = 1/2$}
\\
\\
\centerline{
$\begin{matrix}
X=
\begin{bmatrix}
15/4\\
5/2\\
1/2\\
\end{bmatrix}
\end{matrix}$}
\\
\\
\\
\textbf{Example 2 :: Solve the following system of linear equations :- }
\[
\left\{ 
\begin{array}{c}
x_1 + 4x_2-2x_3 + 8x_4=12 \\ 
x_2-7x_3+2x_4=-4 \\ 
x_3+3x_4=-5\\
x_4=-2
\end{array}
\right. 
\]\\
\\
\\

\textbf{Solution :}\\
\\
\hspace{1.5cm}
\textbf{Step 1 : Write the matrix form of the equation }\\
\\
\\
\centerline{
$\begin{matrix}
\begin{bmatrix}
1 &  4 & -2 & 8\\
0 & 1 &  -7 & 2\\
0 & 0 & 1  & 3\\
0 & 0 & 0 & 1\\
\end{bmatrix}
\end{matrix}$
$\begin{matrix}
\begin{bmatrix}
x_1\\
x_2\\
x_3\\
x_4\\
\end{bmatrix}
\end{matrix}$
$\begin{matrix}
=
\begin{bmatrix}
12\\
-4\\
-5\\
-2\\
\end{bmatrix}
\end{matrix}$}\\ \\ \\

\centerline{
$\begin{matrix}
X=
\begin{bmatrix}
x_1\\
x_2\\
x_3\\
x_4
\end{bmatrix}
\end{matrix}$
}

\textbf{Step 2 : Convert it into agumented matrix }\\
\\

\[
 \left[\begin{array}{rrrr|r}
1 &  4 & -2 & 8 & 12\\
0 & 1 &  -7 & 2 & -4\\
0 & 0 & 1  & 3 & -5\\
0 & 0 & 0 & 1 & -2\\
   \end{array} \right]
\]\\
\\
\textbf{Step 3 : Solve it using various row elementary operations }\\
\\
\\
\[
\left[\begin{array}{rrrr|r}
1 &  4 & -2 & 8 & 12\\
0 & 1 &  -7 & 2 & -4\\
0 & 0 & 1  & 3 & -5\\
0 & 0 & 0 & 1 & -2\\
  \end{array}\right]
  \]
  \\
  \\
  \[
  \xrightarrow{\substack{R_1 \rightarrow R_1 - 8R_4 \\ R_2 \rightarrow R_2 - 2R_4 \\ R_3 \rightarrow R_3 - 3R_4}}
  \left[\begin{array}{rrrr|r}
1 & 4 & -2 & 0 & 28\\
0 & 1 & -7 & 0 &  0\\
0 & 0 &  1 & 0 &  1\\
0 & 0 &  0 & 1 & -2\\
  
  \end{array}\right]
  \]
  \\
  \\
  \
  \[
 \xrightarrow{\substack{R_1 \rightarrow R_1 + 2 R_3 \\ R_2 \rightarrow R_2 + 7R_3}}
\left[\begin{array}{rrrr|r}
1 & 4 &  0 & 0 & 30\\
0 & 1 &  0 & 0 &  7\\
0 & 0 &  1 & 0 &  1\\
0 & 0 &  0 & 1 & -2\\
  \end{array}\right]
  \]\\
  \\
  \[
 \xrightarrow{\substack{R_1 \rightarrow R_1 - 4R_2}}
\left[\begin{array}{rrrr|r}
1 & 0 &  0 & 0 &  2\\
0 & 1 &  0 & 0 &  7\\
0 & 0 &  1 & 0 &  1\\
0 & 0 &  0 & 1 & -2\\
  \end{array}\right]
  \]\\
  \\
  \\
  
\textbf{Step 4 :Equate each row  }\\
\\
\textrm{Here rank of agumented matrix is same as compared to original matrix , therefore it has unique solution.}
\\
\\
\textbf{Solution is} 
\\
\\
\centerline{$ x_1 = 2 $}\\
\centerline{$ x_2 = 7 $}\\
\centerline{$  x_3 = 1$}\\
\centerline{$  x_4 = -2$}
\\
\\
\centerline{
$\begin{matrix}
X=
\begin{bmatrix}
2\\
7\\
1\\
-2\\
\end{bmatrix}
\end{matrix}$}
\\
\\
\\
\textbf{Example 3 :: Solve the following system of linear equations :- }
\[
\left\{ 
\begin{array}{c}
2x_1+x_2-x_3=7 \\ 
4x_1+2x_2-2x_3=14\\ 
4x_1+3x_2-1x_3=5
\end{array}
\right. 
\]
\textbf{Solution :}\\
\\
\hspace{1.5cm}
\textbf{Step 1 : Write the matrix form of the equation }\\
\\
\\
\centerline{
$\begin{matrix}
\begin{bmatrix}
2 &  1 & -1\\
4 &  2 &  -2\\
4 &  3 & -1\\
\end{bmatrix}
\end{matrix}$
$\begin{matrix}
\begin{bmatrix}
x_1\\
x_2\\
x_3\\
\end{bmatrix}
\end{matrix}$
$\begin{matrix}
=
\begin{bmatrix}
7\\
14\\
5\\
\end{bmatrix}
\end{matrix}$}\\ \\ \\

\centerline{
$\begin{matrix}
X=
\begin{bmatrix}
x_1\\
x_2\\
x_3\\
\end{bmatrix}
\end{matrix}$}
$\\
\\
\\
$
\textbf{Step 2 : Convert it into agumented matrix }
\\

\[
 \left[\begin{array}{rrr|r}
2 &  1 & -1 &  7\\
4 &  2 & -2 & 14\\
4 &  3 & -1 &  5\\
   \end{array} \right]
\]\\
\\
\textbf{Step 3 : Solve it using various row elementary operations }\\
\\
\\
\[
\left[\begin{array}{rrr|r}
 2 &  1 & -1 &  7\\
4 &  2 & -2 & 14\\
4 &  3 & -1 &  5\\
  \end{array}\right]
  \]
  \\
  \\
  \[
  \xrightarrow{\substack{R_2 \rightarrow R_2 - 2R_1 \\ R_3 \rightarrow R_3 - 2R_1}}
  \left[\begin{array}{rrr|r}
 2 &  1 & -3 & 7 \\
 0 &  0 &  0 & 0 \\
 0 &  1 &  1 & -9 \\ 
  
  \end{array}\right]
  \]
  \\
  \\
  \
  \[
 \xrightarrow{\substack{R_1 \rightarrow R_1 - R_3}}
\left[\begin{array}{rrr|r}
  2 & 0 & -4 & 16 \\
  0 & 0 &  0 & 0 \\
  0 & 1 &  1 & -9 \\
  \end{array}\right]
  \]\\
  \\
  \[
 \xrightarrow{\substack{R_1 \rightarrow R1/2 \\ R_2 \leftrightarrow R_3}}
\left[\begin{array}{rrr|r}
 1 & 0 & -2 & 8 \\
 0 & 1 &  1 & -9 \\
 0 & 0 &  0 & 0 \\

  \end{array}\right]
  \]\\
  \\
  \\
\textbf{Here, 3rd row is complete zero. \\
\\
\centerline{$ 0x_1 + 0x_2 + 0x_3 =0 $}}\\

\textbf{Step 4 :Equate each row  }\\
\\
\textbf{}{Here rank of agumented matrix is not same as compared to original matrix , therefore it has infinte many solution.}
\\
\\
\textbf{ $x_3$ is a free variable . So all the solutions will be in the term of $x_3.$}\\
\\
\textbf{Obtained equations are :-}\\
\\
\centerline{$x_1 - 2x_3 = 8 \\ x_2 +x_3 = -9$}
\\
\\
\centerline{$ x_1 = 8 + 2x_3 $}\\
\centerline{$ x_2 = -9 - x_3 $}\\
\centerline{$  x_3 = x_3$}
\\
\\
\\
\centerline{
$\begin{matrix}
X=\space x_3
\begin{bmatrix}
 2\\
-1\\
 1\\
\end{bmatrix}
\end{matrix}$
$\begin{matrix}
+
\begin{bmatrix}
 8\\
-9\\
 0\\
\end{bmatrix}
\end{matrix}$}
\\
\\
\\
\\
\textbf{Example 4 :: Solve the following system of linear equations :- }
\[
\left\{ 
\begin{array}{c}
2x_1+x_2-x_3=7 \\ 
4x_1+2x_2-2x_3=15\\ 
4x_1+3x_2-1x_3=5
\end{array}
\right. 
\]
\textbf{Solution :}\\
\\
\hspace{1.5cm}
\textbf{Step 1 : Write the matrix form of the equation }\\
\\
\\
\centerline{
$\begin{matrix}
\begin{bmatrix}
2 &  1 & -1\\
4 &  2 &  -2\\
4 &  3 & -1\\
\end{bmatrix}
\end{matrix}$
$\begin{matrix}
\begin{bmatrix}
x_1\\
x_2\\
x_3\\
\end{bmatrix}
\end{matrix}$
$\begin{matrix}
=
\begin{bmatrix}
7\\
15\\
5\\
\end{bmatrix}
\end{matrix}$}\\ \\ \\

\centerline{
$\begin{matrix}
X=
\begin{bmatrix}
x_1\\
x_2\\
x_3\\
\end{bmatrix}
\end{matrix}$}
$\\
\\
\\
$
\textbf{Step 2 : Convert it into agumented matrix }
\\

\[
 \left[\begin{array}{rrr|r}
2 &  1 & -1 &  7\\
4 &  2 & -2 & 15\\
4 &  3 & -1 &  5\\
   \end{array} \right]
\]\\
\\
\textbf{Step 3 : Solve it using various row elementary operations }\\
\\
\\
\[
\left[\begin{array}{rrr|r}
 2 &  1 & -1 &  7\\
4 &  2 & -2 & 15\\
4 &  3 & -1 &  5\\
  \end{array}\right]
  \]
  \\
  \\
  \[
  \xrightarrow{\substack{R_2 \rightarrow R_2 - 2R_1 \\ R_3 \rightarrow R_3 - 2R_1}}
  \left[\begin{array}{rrr|r}
 2 &  1 & -3 & 7 \\
 0 &  0 &  0 & 1 \\
 0 &  1 &  1 & -9 \\ 
  
  \end{array}\right]
  \]
  \\
  \\
  \
  \[
 \xrightarrow{\substack{R_1 \rightarrow R_1 - R_3}}
\left[\begin{array}{rrr|r}
  2 & 0 & -4 & 16 \\
  0 & 0 &  0 & 0 \\
  0 & 1 &  1 & -9 \\
  \end{array}\right]
  \]\\

  \[
 \xrightarrow{\substack{R_1 \rightarrow R1/2 \\ R_2 \leftrightarrow R_3}}
\left[\begin{array}{rrr|r}
 1 & 0 & -2 & 8 \\
 0 & 1 &  1 & -9 \\
 0 & 0 &  0 & 1 \\

  \end{array}\right]
  \]\\
	\textbf{Here, 3rd row is not complete zero.There is some value left in agumented part after performing row-elementary row operations in matrix. \\
\centerline{$ 0x_1 + 0x_2 + 0x_3 =1 $}}
$\\$
\textbf{Step 4 :Check rows }\\
\textbf{Here rank of agumented matrix is not same as compared to original matrix , therefore it has infinte many solution.}\\
\\
\textbf{ NO solution can be possible.}\\
\\
\\
\begin{Verbatim}[fontsize=\Huge]
------------------- END-------------------
\end{Verbatim}
\end{document}







